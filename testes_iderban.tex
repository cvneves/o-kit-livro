%\documentclass[a4paper, 12pt, oneside]{amsbook}
\documentclass[a4paper,oneside,12pt]{book}
\usepackage{setspace}
\usepackage{float}

%% PROFUNDIDADE DO SUMARIO %%%%
\setcounter{tocdepth}{2}

%%%% MISC %%%%%
\usepackage[useregional]{datetime2}

%%%%%%%%% TABELAS %%%%%%%%%
\usepackage[table]{xcolor}
%\usepackage{tablestyles}
\usepackage{tabularx}
\usepackage{booktabs}
\usepackage{multirow}
\usepackage{xcolor}
%%%%%%%%%%%%%%%%%%%%%%%%%%%

%%%%%%%%%%%%%%%%%%%%%%%%%%%%%%%

%%%%%% MARGENS %%%%%%
%\usepackage{anysize}
\sloppy
%\setstretch{1.125}
\usepackage{geometry}
%\usepackage[headheight=16pt, vmargin=1in,hmargin=1in]{geometry}
%\stretch{1}
\usepackage{marginnote}
\usepackage[bottom]{footmisc}

%%%%%%%%%%%%%%%%%%%%%%%%%%%%%%

%%%%%% FONTES %%%%%%%%%
%\usepackage[light,math]{anttor}
\usepackage[T1]{fontenc}

%\renewcommand*\familydefault{\sfdefault} %%
%\usepackage{sfmath}
%\usepackage[usenames,dvipsnames]{xcolor}

%\usepackage{tgbonum} %% <---- Fonte teen ("How do you do fellow kids?")
%\usepackage{tgschola}
%\usepackage{helvet}
%\fontfamily{qcr}\selectfont
%\usepackage{lmodern}

%%%%%%%%%%%%%%%%%%%%%%

%%%%%%% CONTADORES %%%%%
\makeatletter
\def\providecounter#1{%
  \@ifundefined{c@#1}%
    {\newcounter{#1}}{\@newctr}}
\makeatother

\usepackage{enumitem}% http://ctan.org/pkg/enumitem
%%%%%%%%%%%%%%%%%%%%%%

%%%%%% links %%%%%%%%%%
\usepackage{hyperref}
\usepackage{url}
\hypersetup{
    colorlinks=true,
    linkcolor=blue,
    % filecolor=magenta,      
    % urlcolor=cyan,
    % pdftitle={Overleaf Example},
    % pdfpagemode=FullScreen,
    }
%%%%%%%%%%%%%%%%%%%%%%%

%%%%%%% Português %%%%%%%%
\usepackage[utf8]{inputenc}
\usepackage[brazil]{babel}
%%%%%%%%%%%%%%%%%%%%%%%%%

%%%%%%% Condicionais %%%%%%%
\usepackage{xifthen}
%%%%%%%%%%%%%%%%%%%%%%%%%%%

%%%%%%% Figuras e Tikz %%%%%%%
\usepackage{incgraph}
\usepackage{tikz}
\usepackage{fp} % for rounding

% \usetikzlibrary{external}
% \tikzexternalize % activate!

\usepackage{graphicx}
\usetikzlibrary{shapes.geometric, arrows, arrows.meta} % diamonds e outras formas
\usetikzlibrary{decorations.pathreplacing}
\usetikzlibrary{decorations.pathmorphing}
\usetikzlibrary{bending}
\usetikzlibrary{positioning}
\usetikzlibrary{tikzmark}
\usetikzlibrary{calc}
\usetikzlibrary{patterns}
\usetikzlibrary{fadings}
% \usetikzlibrary{patterns.meta}

\usepackage{dot2texi}

\usepackage{framed}
\usepackage{caption}
\usepackage{subcaption}
%\usepackage{subfig}
%%%%%%%%%%%%%%%%%%%%%%%%%%%


%%%%%% Matemática %%%%%%%
\usepackage{amsmath, amssymb}
\usepackage{amsthm} %% teoremas
\newtheorem{theorem}{Teorema}
\newtheorem{prop}{Proposição}
%%%%%%%%%%%%%%%%%%%%%%%%%%

%%%%% ALGORITMOS %%%%%%%
\usepackage{mathtools}
\usepackage[ruled, vlined, linesnumbered, portuguese,onelanguage]{algorithm2e}
%%%%%%%%%%%%%%%%%%%%%%

%%%%%%%% CÓDIGO %%%%%%%%
\usepackage{listings}
\renewcommand{\lstlistingname}{Código}
\usepackage[scaled=0.9]{DejaVuSansMono}
\lstdefinestyle{cplusplusListStyle}{
  % backgroundcolor=\color{blue!10},
  belowcaptionskip=1\baselineskip,
  breaklines=true,
  xleftmargin=\parindent,
  language=C++,
  % frame=lrtb,
  frame=tb,
  showstringspaces=false,
  tabsize=2,
  %basicstyle=\footnotesize,
  basicstyle=\footnotesize\ttfamily,
  keywordstyle=\bfseries\color{green!40!black},
  commentstyle=\itshape\color{purple!40!black},
  identifierstyle=\color{blue},
  stringstyle=\color{orange},
numbers=left,
stepnumber=1
}
%%%%%%%%%%%%%%%%%%%%%%

%%%% headers %%%%% %%%%%
\usepackage[fit]{truncate}
\usepackage{fancyhdr}
\pagestyle{headings}
\pagestyle{fancy}
\fancyhf{}
\fancyfoot{}
%\rhead{\rightmark}
%\lhead{\truncate{.5\headwidth}{\leftmark}}
\chead{\leftmark}
% \rhead{\thepage}
\renewcommand{\headrulewidth}{0pt}
%\lhead{\leftmark}
\cfoot{\thepage}
%%%%%%%%%%%%%%%%%%%%%%%%

%%%%%%%%%%% CHAPTER STYLE %%%%%%%%%%%
\usepackage{titlesec}
\titleformat{\chapter}[display]
  {\normalfont\bfseries\centering}{\Huge\textbf{\fontsize{70}{80}\selectfont\thechapter}\vspace{0.25cm}}{0pt}{\LARGE}
%%%%%%%%%%%%%%%%%%%%%%%%

%%%%%%%%% COMANDOS %%%%%%%%%%% 
\renewcommand{\chaptermark}[1]{\markboth{\textit{#1}}{}}
\newcommand{\referencia}{{\color{blue} REF}}
%%%%%%%%%%%%%%%%%%%%%%%%

\title{O Kit}
%\author{}
%\date{2020}

\begin{document}

\begin{figure}
\centering
\begin{subfigure}{0.45\textwidth} % Adjust the width as needed
\begin{tikzpicture}
  % Nodes
  \foreach \i in {0,...,10} {
    \node[circle, draw, fill=black, inner sep = 0.1cm] (node\i) at ({360/8 * (\i)}:2.9) {};
  }

  % Comment out the Contorno
  % \draw[opacity=0.8, dashed, line width=1pt, fill opacity=0.5] plot[smooth cycle] coordinates{ (-3.3, 0.4) (2.6, -2) (0, -3.3) (-2.1, -2.7) };
  % \draw[opacity=0.8, dashed, line width=1pt, fill opacity=0.5] plot[smooth cycle] coordinates{ (3.3, -0.4) (-2.6, 2) (0, 3.3) (2.1, 2.7) };

  \draw (node0) -- (node1) node[midway, fill=white, minimum size = 0.1cm, inner sep = 2.05pt, rounded corners = 1pt]  {\footnotesize $1$};
\draw (node0) -- (node2) node[midway, fill=white, minimum size = 0.1cm, inner sep = 2.05pt, rounded corners = 1pt]  {\footnotesize $1$};
\draw (node1) -- (node2) node[midway, fill=white, minimum size = 0.1cm, inner sep = 2.05pt, rounded corners = 1pt]  {\footnotesize $0.75$};
\draw (node1) -- (node3) node[midway, fill=white, minimum size = 0.1cm, inner sep = 2.05pt, rounded corners = 1pt]  {\footnotesize $0.25$};
\draw (node2) -- (node3) node[midway, fill=white, minimum size = 0.1cm, inner sep = 2.05pt, rounded corners = 1pt]  {\footnotesize $0.25$};
\draw (node3) -- (node5) node[midway, fill=white, minimum size = 0.1cm, inner sep = 2.05pt, rounded corners = 1pt, pos=0.5]  {\footnotesize $0.75$};
\draw (node3) -- (node7) node[midway, fill=white, minimum size = 0.1cm, inner sep = 2.05pt, rounded corners = 1pt]  {\footnotesize $0.75$};
\draw (node4) -- (node5) node[midway, fill=white, minimum size = 0.1cm, inner sep = 2.05pt, rounded corners = 1pt]  {\footnotesize $1$};
\draw (node4) -- (node6) node[midway, fill=white, minimum size = 0.1cm, inner sep = 2.05pt, rounded corners = 1pt]  {\footnotesize $0.75$};
\draw (node4) -- (node7) node[midway, fill=white, minimum size = 0.1cm, inner sep = 2.05pt, rounded corners = 1pt]  {\footnotesize $0.25$};
\draw (node5) -- (node6) node[midway, fill=white, minimum size = 0.1cm, inner sep = 2.05pt, rounded corners = 1pt]  {\footnotesize $0.25$};
\draw (node6) -- (node7) node[midway, fill=white, minimum size = 0.1cm, inner sep = 2.05pt, rounded corners = 1pt]  {\footnotesize $1$};

\end{tikzpicture}
\caption{Without Contorno}
\end{subfigure}
\hfill
\begin{subfigure}{0.45\textwidth} % Adjust the width as needed
\begin{tikzpicture}
% Contorno
% \draw[fill=black!20, draw opacity = 0, line width=1pt, fill opacity=0.5] plot[smooth cycle] coordinates{ (-3.3, 0.4) (2.6, -2) (0, -3.3) (-2.1, -2.7) };
% \draw[fill=black!20, draw opacity = 0, line width=1pt, fill opacity=0.5] plot[smooth cycle] coordinates{ (3.3, -0.4) (-2.6, 2) (0, 3.3) (2.1, 2.7) };

  % Nodes
  \foreach \i in {0,...,10} {
    \node[circle, draw, fill=black, inner sep = 0.1cm] (node\i) at ({360/8 * (\i)}:2.9) {};
  }

\draw[dashed, gray] (node0) -- (node1); % node[midway, fill=white, minimum size = 0.1cm, inner sep = 2.05pt, rounded corners = 1pt]  {\footnotesize $1$};
\draw[dashed, gray] (node0) -- (node2); % node[midway, fill=white, minimum size = 0.1cm, inner sep = 2.05pt, rounded corners = 1pt]  {\footnotesize $1$};
\draw[dashed, gray] (node1) -- (node2); % node[midway, fill=white, minimum size = 0.1cm, inner sep = 2.05pt, rounded corners = 1pt]  {\footnotesize $0.75$};
\draw[dashed, gray] (node1) -- (node3); % node[midway, fill=white, minimum size = 0.1cm, inner sep = 2.05pt, rounded corners = 1pt]  {\footnotesize $0.25$};
\draw[dashed, gray] (node2) -- (node3); % node[midway, fill=white, minimum size = 0.1cm, inner sep = 2.05pt, rounded corners = 1pt]  {\footnotesize $0.25$};

\draw[] (node3) -- (node5) node[midway, fill=white, minimum size = 0.1cm, inner sep = 2.05pt, rounded corners = 1pt, pos=0.5]  {\footnotesize $0.75$};

\draw[] (node3) -- (node7) node[midway, fill=white, minimum size = 0.1cm, inner sep = 2.05pt, rounded corners = 1pt]  {\footnotesize $0.75$};

\draw[dashed, gray] (node4) -- (node5); %node[midway, fill=white, minimum size = 0.1cm, inner sep = 2.05pt, rounded corners = 1pt]  {\footnotesize $1$};
\draw[dashed, gray] (node4) -- (node6); %node[midway, fill=white, minimum size = 0.1cm, inner sep = 2.05pt, rounded corners = 1pt]  {\footnotesize $0.75$};
\draw[dashed, gray] (node4) -- (node7); %node[midway, fill=white, minimum size = 0.1cm, inner sep = 2.05pt, rounded corners = 1pt]  {\footnotesize $0.25$};
\draw[dashed, gray] (node5) -- (node6); %node[midway, fill=white, minimum size = 0.1cm, inner sep = 2.05pt, rounded corners = 1pt]  {\footnotesize $0.25$};
\draw[dashed, gray] (node6) -- (node7); %node[midway, fill=white, minimum size = 0.1cm, inner sep = 2.05pt, rounded corners = 1pt]  {\footnotesize $1$};

\draw[dashed, line width = 2pt] plot[smooth, tension=0.5] coordinates { (-3.25, 3) (-1.5, 0.5) (1.5, 0.5) (3.25, -3) };

\end{tikzpicture}
\caption{With Contorno}
\end{subfigure}

\caption{Two Subfigures}
\end{figure}


% \begin{tikzpicture}
%   % Nodes
%   \foreach \i in {0,...,10} {
%     \node[circle, draw, fill=black, inner sep = 0.1cm] (node\i) at ({360/8 * (\i)}:2.9) {};
%   }

% % contorno
% \draw[opacity=0.8, dashed, line width=1pt, fill opacity=0.5] plot[smooth cycle] coordinates{ (-3.3, 0.4) (2.6, -2) (0, -3.3) (-2.1, -2.7) };
% opacity=0.5, 
% \draw[opacity=0.8, dashed, line width=1pt, fill opacity=0.5] plot[smooth cycle] coordinates{ (3.3, -0.4) (-2.6, 2) (0, 3.3) (2.1, 2.7) };
% % contorno


% % \draw[pattern=dots, line width=1pt, fill opacity=0.5, rounded corners = 1cm] (-3.6, 0.5) -- (2.8, -2) -- (0, -3.3) -- (-2.1, -2.7)  -- cycle;
 
% \draw (node0) -- (node1) node[midway, fill=white, minimum size = 0.1cm, inner sep = 2.05pt, rounded corners = 1pt]  {\footnotesize $1$};
% \draw (node0) -- (node2) node[midway, fill=white, minimum size = 0.1cm, inner sep = 2.05pt, rounded corners = 1pt]  {\footnotesize $1$};
% \draw (node1) -- (node2) node[midway, fill=white, minimum size = 0.1cm, inner sep = 2.05pt, rounded corners = 1pt]  {\footnotesize $0.75$};
% \draw (node1) -- (node3) node[midway, fill=white, minimum size = 0.1cm, inner sep = 2.05pt, rounded corners = 1pt]  {\footnotesize $0.25$};
% \draw (node2) -- (node3) node[midway, fill=white, minimum size = 0.1cm, inner sep = 2.05pt, rounded corners = 1pt]  {\footnotesize $0.25$};
% \draw (node3) -- (node5) node[midway, fill=white, minimum size = 0.1cm, inner sep = 2.05pt, rounded corners = 1pt, pos=0.35]  {\footnotesize $0.75$};
% \draw (node3) -- (node7) node[midway, fill=white, minimum size = 0.1cm, inner sep = 2.05pt, rounded corners = 1pt]  {\footnotesize $0.75$};
% \draw (node4) -- (node5) node[midway, fill=white, minimum size = 0.1cm, inner sep = 2.05pt, rounded corners = 1pt]  {\footnotesize $1$};
% \draw (node4) -- (node6) node[midway, fill=white, minimum size = 0.1cm, inner sep = 2.05pt, rounded corners = 1pt]  {\footnotesize $0.75$};
% \draw (node4) -- (node7) node[midway, fill=white, minimum size = 0.1cm, inner sep = 2.05pt, rounded corners = 1pt]  {\footnotesize $0.25$};
% \draw (node5) -- (node6) node[midway, fill=white, minimum size = 0.1cm, inner sep = 2.05pt, rounded corners = 1pt]  {\footnotesize $0.25$};
% \draw (node6) -- (node7) node[midway, fill=white, minimum size = 0.1cm, inner sep = 2.05pt, rounded corners = 1pt]  {\footnotesize $1$};

% \end{tikzpicture}

\end{document}

